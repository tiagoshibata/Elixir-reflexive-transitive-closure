\documentclass[conference]{IEEEtran}
\usepackage[utf8]{inputenc}
\usepackage{amsfonts}
\usepackage{svg}
\usepackage{listings}
\lstset{
    basicstyle=\tiny\ttfamily,
    keywordstyle=\color{blue}\ttfamily,
    stringstyle=\color{red}\ttfamily,
    commentstyle=\color{green}\ttfamily,
    breaklines=true
}
% *** PDF, URL AND HYPERLINK PACKAGES ***
%
%\usepackage{url}
% url.sty was written by Donald Arseneau. It provides better support for
% handling and breaking URLs. url.sty is already installed on most LaTeX
% systems. The latest version and documentation can be obtained at:
% http://www.ctan.org/pkg/url
% Basically, \url{my_url_here}.

% correct bad hyphenation here
\hyphenation{op-tical net-works semi-conduc-tor}

\begin{document}
\title{Relatório - EP1}

% author names and affiliations
% use a multiple column layout for up to three different
% affiliations
\author{\IEEEauthorblockN{Tiago K C Shibata}
\IEEEauthorblockA{Escola Politécnica\\
Universidade de São Paulo\\
tiago.shibata@usp.br}
}

\maketitle

% For peerreview papers, this IEEEtran command inserts a page break and
% creates the second title. It will be ignored for other modes.
\IEEEpeerreviewmaketitle

\section{Introdução}
Esse relatório acompanha o primeiro exercício programa (EP1) da disciplina PCS3556 - Lógica Computacional.

\hfill 11 de Fevereiro de 2018

\section{Tarefa}

A tarefa consiste em implementar o algoritmo de fecho reflexivo e transitivo de uma relação binária $R \subseteq A X A$ sobre um conjunto finito A que é descrita por meio de um grafo direcionado, usando recursão em uma linguagem funcional (Elixir). A entrada e saída são representações do grafo como lista de tuplas (lista de arestas representadas por pares indicando origem e destino).

\section{Estruturas de dados}

No código, a lista de tuplas foi convertida para mapa de adjacências. Cada chave do mapa indica um vértice de origem. O valor da chave é um conjunto ($MapSet$). A escolha das estruturas de dados foi feita tendo em mente performance e facilidade: o uso do mapa evita que varramos toda a lista toda vez que formos buscar adjacências, e o conjunto permite fácil e rápida únião de conjuntos quando desejamos adicionar arestas.

\section{Algoritmo}

Para buscar todos os vértices alcançáveis a partir de uma origem (fecho transitivo), foi usada uma busca por profundidade em cima do mapa de adjacências. O algoritmo é bastante conhecido e dispensa apresentações.

\section{Código e testes}

As funções para conversão de lista de tuplas para mapas de adjacências foram implementadas e testadas:

\lstinputlisting{1.ex}

Os testes foram desenvolvidos junto com o código:

\lstinputlisting{1_test.ex}

Uma função de busca por profundidade a partir de um vértice foi implementada, usando recursão:

\lstinputlisting{2.ex}

E uma função para chamar $dfs\_from\_vertex$ para todos os vértices e retornar um mapa de conjuntos de vértices alcançáveis:

\lstinputlisting{3.ex}

Funções de busca por profundidade foram testadas:

\lstinputlisting{3_test.ex}

Por fim, foi feita e testada a conversão final, de mapa de adjacências para lista de pares:

\lstinputlisting{4.ex}

Testes:

\lstinputlisting{4_test.ex}

\section{Créditos}
Agradeço ao professor Ricardo Luis de Azevedo da Rocha pelo conhecimento transmitido.

\begin{thebibliography}{1}
\bibitem{infinormatica}
Friedel Ziegelmayer. \emph{Elixir ExDoc}. https://hexdocs.pm/elixir/, acessado em 11/02/2018
\end{thebibliography}

\end{document}
